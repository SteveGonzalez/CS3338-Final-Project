\documentclass[11pt]{article}

% Packages
\usepackage[a4paper,margin=1in]{geometry}
\usepackage{hyperref}
\usepackage{parskip}
\usepackage{titlesec}

%----------------------------------------------------------------------------------

% Formatting tweaks
\titleformat{\section}{\large\bfseries}{\thesection}{1em}{}
\titleformat{\subsection}{\normalsize\bfseries}{\thesubsection}{1em}{}

\title{User Manual}
\author{Group 2} 
\date{December 2025}

\begin{document}
\maketitle

%----------------------------------------------------------------------------------

\section{JIRA Link}
The project tasks, progress tracking, and team coordination were managed through Jira.  
Jira was used to assign responsibilities, track feature development, and document progress throughout the project lifecycle.

\textbf{Link:} \\
\url{https://cs3338-group-2-2025.atlassian.net/jira/software/projects/MTAR/boards/105}
%----------------------------------------------------------------------------------

\section{Features}
The Moon Augmented Reality project is a conceptual senior design project inspired by real-world AR astronomy tools.  
The following features describe the intended functionality of the system:

\begin{itemize}
    \item Augmented reality overlays that display celestial objects such as the Moon, stars, and constellations when viewed through a telescope.
    \item Visual guides to assist users with telescope alignment and object tracking.
    \item Informational labels and annotations that provide educational context about observed celestial bodies.
    \item A simple and intuitive user interface designed to be accessible for both beginners and experienced users.
    \item Real-time synchronization between the telescope’s orientation and the AR display.
\end{itemize}

These features focus on improving accessibility, learning, and usability in amateur astronomy.

%----------------------------------------------------------------------------------

\section{Objectives}

\begin{itemize}
    \item Provide AR overlays that identify celestial bodies, constellations, and key astronomical features in real time.
    \item Assist users in telescope alignment and targeting through visual markers and guided instructions.
    \item Implement precise spatial tracking to synchronize AR visuals with the telescope’s orientation and movement.
    \item Enhance astronomy education by displaying contextual information, object details, and live annotations.
    \item Develop an intuitive user interface that prioritizes ease of use and clarity.
\end{itemize}

%----------------------------------------------------------------------------------

\section{Access Application}
At this stage, the MoonTrek AR application is a prototype and is not publicly deployed. The application is intended to be accessed using an AR-capable device connected to a telescope setup. During development, the app would be run locally by the team to test features such as object overlays, alignment guidance, and educational information. Public access and deployment are planned for future development.

%----------------------------------------------------------------------------------

\end{document}